%%%%%%%%%%%%%%%%%%%%%%%%%%%%%%%%%%%%%%%%%
% Keros Resume/CV
% XeLaTeX Template
% Version 1.0 (5/5/13)
%
% This template has been downloaded from:
% http://www.LaTeXTemplates.com
%
% Original author:
% Adrien Friggeri (adrien@friggeri.net)
% https://github.com/afriggeri/CV
%
% License:
% CC BY-NC-SA 3.0 (http://creativecommons.org/licenses/by-nc-sa/3.0/)
%
% Important notes:
% This template needs to be compiled with XeLaTeX and the bibliography, if used,
% needs to be compiled with biber rather than bibtex.
%
%%%%%%%%%%%%%%%%%%%%%%%%%%%%%%%%%%%%%%%%%

\documentclass[print]{keros-cv} % Add 'print' as an option into the square bracket to remove colors from this template for printing

%\addbibresource{bibliography.bib} % Specify the bibliography file to include publications

\begin{document}

\header{Αλέξανδρος Δημήτριος}{ Κέρος}{} % Your name and current job title/field

%----------------------------------------------------------------------------------------
%	SIDEBAR SECTION
%----------------------------------------------------------------------------------------

\begin{aside} % In the aside, each new line forces a line break
\section{Επικοινωνία}
Παπαναστασίου 4Β
Μελίσσια, Αθήνα 15127
Ελλάδα
~
+0 (030) 6944858616
+0 (030) 2108032049
~
\href{mailto:alexdkeros@gmail.com}{alexdkeros@gmail.com}
\section{Γλώσσες}
Ελληνικά (μητρική γλώσσα)
Αγγλικά (ευχέρεια)
Γερμανικά (B2 Mittelstufe Deutsch)
\end{aside}

%----------------------------------------------------------------------------------------
%	EDUCATION SECTION
%----------------------------------------------------------------------------------------

\section{Εκπαίδευση και Κατάρτιση}

\begin{entrylist}
%------------------------------------------------
\entry
{2008-- }
{Σχολή Ηλεκτρονικών Μηχανικών και Μηχανικών Ηλεκτρονικών Υπολογιστών}
{Πολυτεχνείο Κρήτης}
{(Τρέχων Μέσος Όρος: 7.20, Μόρια Εισαγωγής: 17.662, Σειρά Εισαγωγής: 12η)
Εκπόνηση Διπλωματικής Εργασίας υπό την επίβλεψη του κ. Βασίλη Σαμολαδά με θέμα: \emph{Scaling Geometric Monitoring over Distributed Streams}\\}
%------------------------------------------------
\entry
{--2008}
{Γενικό Λύκειο}
{Τοσίτσειο-Αρσάκειο Εκάλης}
{Τεχνολογική Κατεύθυνση}
%------------------------------------------------
\end{entrylist}

%----------------------------------------------------------------------------------------
%	WORK EXPERIENCE SECTION
%----------------------------------------------------------------------------------------

\section{Επαγγελματική Εμπειρία}

\begin{entrylist}
%------------------------------------------------
\entry
{2013}
{Π. ΜΕΛΙΣΣΑΡΟΠΟΥΛΟΣ ΚΑΙ ΣΙΑ Ο.Ε.}
{Αθήνα, Αττική, Ελλάδα}
{Πρακτική Άσκηση \emph{Junior Web Developer} \\
Απασχόληση σε εταιρεία email marketing (moosend.com) στον τομέα
του web development. Υλοποίηση εφαρμογής social network parser για την συλλογή πληροφοριών από
κοινωνικά δίκτυα (facebook, twitter, linkedin, foursquare, google plus). Υλοποίηση plugins για την
εξυπηρέτηση των απαιτήσεων email marketing (mailing list management, subscriber management,
campaing management) σε πλατφόρμες eshop (shopify, magento etc) και web tools (wordpress, drupal,
etc).}
%------------------------------------------------
\end{entrylist}

%----------------------------------------------------------------------------------------
%   CONFERENCES SECTION
%----------------------------------------------------------------------------------------

\section{Συνέδρια,Σεμινάρια,Εργαστήρια}
\begin{entrylist}
%----------------------------------------------------------------------------------------
\entry
{2015}
{Εργαστήριο κατασκευής Fuzz Pedal από την ομάδα των JAM pedals\\}
{Στέγη Γραμμάτων και Τεχνών Ιδρύματος Ωνάση / Onassis Cultural Centre Athens, Αθήνα, Ελλάδα}
{Κατασκευή fuzz pedal υπό την καθοδήγηση του Γιάννη Αναστασάκη και της ομάδας των JAM pedals.}

%----------------------------------------------------------------------------------------
\entry
{2015}
{Internet of Things: Ευκαιρίες και Κίνδυνοι}
{Ίδρυμα Ευγενίδου, Αθήνα, Ελλάδα}
{Το \emph{MIT Enterprise Forum Greece} και το \emph{MIT Club of Greece}, στα πλαίσια του κύκλου \emph{Technologies That Matter}, διοργάνωσαν την εκδήλωση \emph{Internet of Things:} Ευκαιρίες και Κίνδυνοι, με ομιλητές τους:
\begin{itemize}
\item[-] David Rose, CEO, Ditto Labs, Συγγραφέας, Επιστημονικός Συνεργάτης, MIT Media Lab
\item[-] Άγγελος Μπλέτσας, Αναπληρωτής Καθηγητής, Σχολή Ηλεκτρονικών Μηχανικών \& Μηχανικών Υπολογιστών (Η.Μ.Μ.Υ.), Πολυτεχνείο Κρήτης
\item[-] Δημήτρης Λεονάρδος, VP Product Management, Econais
\item[-] Emilio Frazzoli, Καθηγητής Αεροναυτικής και Αστροναυτικής, MIT
\item[-] Paul Jenkins, EMEA Digital Platform Architect, Oracle
\end{itemize}
}
%----------------------------------------------------------------------------------------
\entry
{2014}
{EASSS 2014}
{Χανιά, Κρήτη, Ελλάδα}
{\emph{16th European Agent Systems Summer School (EASSS 2014)}\\
Εθελοντική συμμετοχή και παρακολούθηση του 16ου Ευρωπαϊκού Θερινού Σχολείου για Συστήματα Πρακτόρων.}
%----------------------------------------------------------------------------------------
\entry
{2014}
{Κωνσταντίνος Δασκαλάκης-Από την πληροφορία στην Πληροφορική\\}
{The HUB Events, Αθήνα, Ελλάδα}
{Διάλεξη του καθηγητή Πληροφορικής στο MIT Κωνσταντίνου Δασκαλάκη με θέμα «Από την πληροφορία στην Πληροφορική», στο πλαίσιο του Hub Science.}
\end{entrylist}

\newpage
%----------------------------------------------------------------------------------------
%   PROGRAMMING LANGUAGES SECTION
%----------------------------------------------------------------------------------------
\section{Προγραμματιστικές Γνώσεις}
\begin{itemize}
\item Γλώσσες Προγραμματισμού και Περιγραφικές Γλώσσες
\begin{itemize}
\item[] Java
\item[] Python
\item[] C
\item[] Matlab
\item[] Javascript
\item[] PHP
\item[] SQL
\item[] Assembly
\item[] VHDL
\item[] HTML \& CSS
\item[] LaTeX
\item[] bash shell
\end{itemize}
\item Λειτουργικά Συστήματα
\begin{itemize}
\item[] Linux (all flavors)
\item[] Windows
\end{itemize}
\item Εργαλεία και Εφαρμογές
\begin{itemize}
\item[] MySQL
\item[] Node.js \& Angular.js
\item[] Apache Tomcat
\item[] Apache Nutch
\item[] Apache Lucene
\item[] Apache Hadoop
\item[] Office Suite
\item[] Eclipse IDE
\end{itemize}
\end{itemize}

%----------------------------------------------------------------------------------------
%	INTERESTS SECTION
%----------------------------------------------------------------------------------------

\section{Ενδιαφέροντα}

\textbf{Επαγγελματικά:} Αλγόριθμοι, Βάσεις Δεδομένων, Ροές Δεδομένων, Τεχνητή Νοημοσύνη, Ανάπτυξη Εφαρμογών, Θεωρία Παιγνίων, Θεωρητικά Μαθηματικά

\textbf{Λοιπά:} Μουσική, Κιθάρα, Συναυλίες, Κινηματογράφος, Βιβλία, Σχέδιο, Πολεμικές Τέχνες

%----------------------------------------------------------------------------------------
%	PUBLICATIONS SECTION
%----------------------------------------------------------------------------------------
%
%\section{publications}
%
%\printbibsection{article}{article in peer-reviewed journal} % Print all articles from the bibliography
%
%\printbibsection{book}{books} % Print all books from the bibliography
%
%\begin{refsection} % This is a custom heading for those references marked as "inproceedings" but not containing "keyword=france"
%\nocite{*}
%\printbibliography[sorting=chronological, type=inproceedings, title={international peer-reviewed conferences/proceedings}, notkeyword={france}, heading=subbibliography]
%\end{refsection}
%
%\begin{refsection} % This is a custom heading for those references marked as "inproceedings" and containing "keyword=france"
%\nocite{*}
%\printbibliography[sorting=chronological, type=inproceedings, title={local peer-reviewed conferences/proceedings}, keyword={france}, heading=subbibliography]
%\end{refsection}
%
%\printbibsection{misc}{other publications} % Print all miscellaneous entries from the bibliography
%
%\printbibsection{report}{research reports} % Print all research reports from the bibliography

%----------------------------------------------------------------------------------------

\end{document}